\section{Conclusion}
\label{sec:conclusion}

We have described three of the earliest networked video games, detailed how they worked, and discussed the historical context they existed in as well as issues that manifested in the implementations.

\textit{Maze} is notable for being the very first networked video game, targeting the IMLAC PDS-1 and using a serial connection and a server to repeat messages from a client to other connected clients.

\textit{Sopwith} is a demo for BMB Compuscience's proprietary Imaginet networking software -- a transparent, operating system-agnostic shared drive which allowed any software to use its network by reading and writing to a virtual floppy drive.

\textit{SGI Dogfight} is significant for being the first video game to use the then-nascent TCP/IP stack which is now the underlying architecture for the internet and networks which interface with it.

Networked video games gave rise to a new means for players to interact with each other, no longer confined to a single display, but free to compete or collaborate from anywhere in the world.

At the time of writing this review, we are closely following the timeline elaborated in the proposal, and we are on track to complete the full work within the given time. The final report will include at least two more games -- \textit{Habitat} (1986) and \textit{Netrek} (1988) -- examined in a similar fashion as above.
