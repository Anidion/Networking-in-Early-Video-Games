\label{subsec:sopwith}

\subsection{Sopwith}

Released in 1984 by Canadian company BMB Compuscience as a demo for Imaginet -- their proprietary networking system -- \textit{Sopwith} is a simple video game where up to four players fly their own Sopwith Camel aircraft from a side-on 2D perspective. A more modern version of the code written by David L. Clark is available and open source, but the original networking components are still present, which we will look at.

\textit{Sopwith} itself does not implement much networking, and heavily relies on the Imaginet protocol, so we will discuss this first. Imaginet was a hardware adapter built by BMB Compuscience which could connect Atari STs, IBM PCs, and other popular computers of the time \cite{colead}. MS-DOS introduced native networking later in 1984 with DOS 3.1, but DOS 2.1 was still sold in alongside it \cite{dosonline}, and IEEE 802.3  was still in its infancy, so Imaginet found a niche in the market. Imaginet was essentially a single floppy drive shared by all connected clients. The host would either have a physical drive or a virtual floppy drive image on a hard disk drive. Clients would have hardware installed in the place of a floppy disk controller, making Imaginet fully transparent to clients, appearing to the computer as a normal floppy drive. This meant it could be compatible with any device that supported floppy drives, regardless of operating system, and any software could become networked by addressing the shared disk. To prevent race conditions when multiple devices try to access the same file simultaneously, a microprocessor in the hardware controls access to the disk, marking it as \mono{BUSY} while in use \cite{imaginetpatent}.

As for \textit{Sopwith}, it makes use of Imaginet using a custom interface called Diskette I/O or DKIO. The implementation is very simple -- if the game is in multiplayer mode, the multiplayer state is saved to and loaded from the virtual floppy drive. The state includes all relevant multiplayer information like player count, max players, state, physics, fuel, score, and physics information of each player. The state in the local memory and the state on the server is synced once per game loop, \ie at frame-rate. An unusual implementation detail is that all clients write to the same specific virtual ``disk sector'' rather than using some kind of file system lookup, an interesting example of the transparency of Imaginet. This would prove problematic if the sector stored other data, overwriting or corrupting it, but since \textit{Sopwith} was intended as a demo, this was likely not a concern.

In more modern times, Imaginet has been emulated by a project called Imaginot \cite{imaginot}. Imaginot works by intercepting and responding to low level DOS system calls, sending the inputs over a specified network protocol, allowing \textit{Sopwith} to be played in multiplayer mode without needing the original Imaginet hardware. Although this is beyond the scope of this project, it is interesting since Imaginot is transparent to Imaginet, which itself is transparent to \textit{Sopwith}, and so running \textit{Sopwith} now requires two layers of networking technology which act as virtual devices, a good example of the challenges in early video game networking before TCP/IP became standard.
