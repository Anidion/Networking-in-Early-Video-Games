\section{Introduction}
\label{sec:intro}

% Background, motivation, problem description, objective, challenges faced, main contributions, and report organization.

Since the very early days of video game history, players have competed or collaborated with each other in multi-player game modes \cite{bnlfirst}. This was achieved by connecting multiple separate input devices or a single shared device to a single computer, and proved effective for players near enough to the computer but inadequate for any distance beyond the length of an input device's cable. It was also infeasible for more than a handful of players to play together. Modern networked video games have solved these problems, allowing hundreds to millions of players to play together simultaneously from anywhere with an internet connection. We will discuss the early steps of the journey to achieve this by looking at three of the earliest networked video games -- with source code available -- from the 1970s and 1980s.

These early video games are \textit{Maze}, \textit{Sopwith}, and \textit{SGI Dogfight}. \textit{Maze} and \textit{Sopwith} were created before the ubiquity of the TCP/IP stack, and \textit{SGI Dogfight} in the very early days of TCP/IP. We will discuss their solutions for networking without reliance on the standard protocols we know and rely on today. This will be achieved through static analysis of the source code and reports from the original developers or players. We aim to emphasize the significance of each game, its unique solution to networking, and any issues that it may have faced.
