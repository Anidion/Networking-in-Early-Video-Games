\section{Preliminary}
\label{sec:pre}

% Knowledge required to understand the report, or formally describing your problem and/or objective in depth

In this review, we will examine the networking implementations used by a variety of early networked video games, assess their implementation details and speculate on their performance and potential flaws. For the purposes of this research, a networked video game may be defined as a video game in which two or more independent clients running the same code interact over a real-time link. Each client therefore handles I/O, networking, and at least some portion of game logic.

Many early video games supported multiplayer functionality, allowing players to compete or co-operate in real time. However, these early solutions were typically multiple input methods connected to a single device with a single display. Although this was effective -- and indeed is still used in contemporary games -- geographically separated players had no means by which to interact.
The PLATO system featured the ability to host multiple graphics terminals (thin clients) which served to simply relay input and output to a mainframe which performed the processing. A number of games were built using it, allowing users to play together even if they were in a different room, and as many mainframes running PLATO were networked together, the terminals were effectively networked together as well \cite{plato}.
Despite the importance of these games, they do not fall under our specifications of a networked video game -- the network code is restricted to the PLATO system on the mainframe itself rather than the client.

Within the scope of our definition, there were some common technologies used by the games we will examine. RS-232 serial connections were used, either as a direct connection between two clients (the serial cable functioning as a null modem), or connecting all clients to a central mainframe which could would broadcast data received from one client to all others. According to the specification, these cables were limited to 20kbps transfer speed \cite{Buchanan2004}, although many were used in unintended ways and specified limits and uses weren't always followed. These cables were effective as null modems when connecting two clients, but this is not part of the RS-232 standard, and had speeds lower than alternatives (like Ethernet), making them not suitable for network transfer beyond simple peer-to-peer two-player implementations.

In 1980, Ethernet was published as an open standard by a partnership between Xerox, DEC, and Intel \cite{DIX} , being standardized by the IEEE in draft form as IEEE 802.3 in 1983 \cite{IEEE8023}. Early Ethernet used a shared cable where all connected devices would receive the same data, resulting in collision risk and shared bandwidth. Despite its limitations, Ethernet was simpler, cheaper, and more effective than its competitors, and it became the \textit{de facto} standard for wired local communication \cite{metcalfe}.

Lastly, Transmission Control Protocol and Internet Protocol (TCP/IP) was made the sole transmission protocol on the ARPANET in January 1983 \cite{flagday}, which persisted as ARPANET transitioned into the public internet, and of course is still in use today.

Before discussing the games, we will elaborate on the methodology used to create the final list. Wikipedia was chosen as the source for finding games due to the amount of data available and well-organized categorization. A Python program (\href{https://github.com/Anidion/wikipedia-scraper}{Source Code available on GitHub}) was written by me, making use of the \mono{pywikibot} library \cite{pywikibot} to pull the \mono{.wikitext} file for every article in the categories "Category:<year> video games" for every year between 1970 and 1989 (inclusive) \ie the article for every video game published between 1970 and 1989. Once this list of nearly four thousand games was extracted, keywords were searched for within the article text by the program searching for at least one reference to a keyword in the text. More complex means of identification were attempted but were found to reject a large number of valid candidates, and so the aforementioned method was used in conjunction with a manual review of the final short-list of just under two hundred video games. This manual review consisted of discarding articles in which the only references to the keywords were unrelated to the game (\eg citations of publications with ``network'' in their name). Sixteen games were left, and of these, five were chosen for their available source and significance. The three games discussed here are the first three in chronological order by release year.
